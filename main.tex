\documentclass[letterpaper,12pt]{article}

\usepackage{graphicx}
\DeclareGraphicsExtensions{.pdf,.png,.gif,.jpg}
\usepackage[usenames,dvipsnames,svgnames,table]{xcolor}
\definecolor{light-gray}{gray}{0.5}
\usepackage{amsmath}
\usepackage{amsthm}
\usepackage[hyperfootnotes=false]{hyperref}
\hypersetup{colorlinks=true,linkcolor=blue,anchorcolor=blue,citecolor=blue,filecolor=blue,urlcolor=blue,bookmarksnumbered=true,pdfview=FitB} %
\usepackage{doi}
\usepackage{tikz}
\usepackage{ulem}
\usepackage{bm}
\usepackage{tabu}
\usepackage{subfig}
\usepackage{listings}
\usepackage[noabbrev,capitalize]{cleveref}
\newcommand{\mc}[1]{\multicolumn{1}{c}{#1}}
\usepackage{siunitx}
\sisetup{
  %use option "per-mode = reciprocal-positive-first" on a command to make it use e.g. m s^{-1}
  %use option "per-mode = symbol" on a command to make it use e.g. m/s
  per-mode = reciprocal-positive-first,
  binary-units = true,
  %range-phrase = {\text{--}},  % or { to }
  range-phrase = { to },  % or { to }
  %range-units = single,  % or repeat or brackets
  range-units = repeat,  % or repeat or brackets
  list-units = single,  % or repeat or brackets
  multi-part-units = single,  % or repeat or brackets
  separate-uncertainty = true, % use \pm instead of brackets
  mode = text,
  %detect-mode = true,
  %detect-shape = true,
  detect-family = true,
  detect-weight = true
}%
% Remember to always put {} after these commands to fix spacing issues!
\newcommand{\GeV}{\giga\electronvolt}
\newcommand{\GeVc}{\GeV{}\per c}
\newcommand{\GeVcc}{\GeV{}\per\square c}
\newcommand{\TeV}{\tera\electronvolt}
\newcommand{\TeVc}{\TeV{}\per c}
\newcommand{\TeVcc}{\TeV{}\per\square c}
\newcommand{\MeV}{\mega\electronvolt}
\newcommand{\MeVc}{\MeV{}\per c}
\newcommand{\MeVcc}{\MeV{}\per\square c}
\newcommand{\keV}{\kilo\electronvolt}
\newcommand{\keVc}{\keV{}\per c}
\newcommand{\keVcc}{\keV{}\per\square c}
\newcommand{\eV}{\electronvolt}
\newcommand{\eVc}{\eV{}\per c}
\newcommand{\eVcc}{\eV{}\per\square c}

\newcommand{\pip}{\ensuremath{\pi^{+}}}
\newcommand{\pim}{\ensuremath{\pi^{-}}}
\newcommand{\pipm}{\ensuremath{\pi^{\pm}}}
\newcommand{\piz}{\ensuremath{\pi^{0}}}
\newcommand{\ep}{\ensuremath{\text{e}^{+}}}
\newcommand{\em}{\ensuremath{\text{e}^{-}}}
\newcommand{\epm}{\ensuremath{\text{e}^{\pm}}}
\newcommand{\gam}{\ensuremath{\gamma}}

\newcommand{\mSqrBeam}{\ensuremath{m^2_{\mathrm{beamline}}}}

%%% Stuff to print git commit in footer
\usepackage[mark]{gitinfo2}

\title{Pion Absorption $+$ Charge Exchange Cross-section with LArIAT}
\author{Justin Hugon, William Metcalf, Andrew Olivier,\\Martin Tzanov, Derek Walker}

\begin{document}

\maketitle

\section{Introduction}

An important part of neutrino interactions is the hadronic recoil.  In
charged-current interactions, a lepton and the hadronic recoil are produced,
while in neutral-current interactions, the only visible sign of the interaction
is the hadronic recoil. In either case, accurately reconstructing the hadronic
recoil is key to accurately estimating neutrino kinematics in a neutrino
detector. The US physics community has decided that liquid argon
time-projection chamber (LArTPC) technology will be the dominant neutrino
detector technology for the coming decades in the MicroBooNE, SBND, and DUNE
experiments. Understanding how these hadrons interact in argon is important
for accurately reconstructing the hadronic recoil in these experiments. Many
studies have been performed on hydrogen, hydrocarbons, and iron for previous
experiments, but argon is not so well studied.  The absorption reaction of
charged pions is particularly important, because a pion may be absorbed in the
neutrino interaction nucleus without leaving a signal in the detector.  Also
important is pion charge exchange, where a charged pion interacts and produces
a neutral pion. Since neutral pions decay to pairs of photons, leading to
electromagnetic showers in the detector, this can make a neutral-current
neutrino interaction look similar to charged-current electron interactions
which also lead to electromagnetic showers.  One of the main missions of LArIAT
is to measure the cross-sections of charged pions on argon.  This note focuses
specifically on the pion absorption and charge exchange pion interactions.

In pion absorption, a charged pion interacts with a nucleus to produce non-pion
hadrons. Protons and neutrons are the most common products of pion absorption,
but deuterons and kaons can also be
produced~\cite{Kotlinski:1998vh,Rowntree:1999dp,Kotlinski:2000hp,Androic:2001tq}.
Neutrons are difficult to detect in a LArTPC because, as neutral particles,
they only produce ionization electrons when they interact with nuclei. Thus,
observations of pion absorption in argon rely on identification of protons,
deuterons, and very rarely kaons.

In pion charge exchange, a charged pion interacts with a nucleus so that the
pion's charge is added to the nucleus and a neutral pion is produced. Neutral
pions decay to a pair of photons, and the short neutral pion lifetime means
that they decay near the nucleus where they were produced. Thus, charge
exchange interactions are similar to pion absorption interactions, but with the
addition of two electromagnetic showers.

\section{Data \& Monte Carlo Samples}

Data for this study comes from LArIAT Run II which took place in 2016. For the
main analysis, only the positive magnet polarity runs are used. Some
cross-checks use the negative polarity runs. The data considered are from the
SAM dataset definitions \texttt{Lovely1\_Pos\_RunII\_elenag\_v04} and
\texttt{Lovely1\_Neg\_RunII\_elenag\_v04} which contain 57219 and 72532 files,
respectively. ``Lovely1'' is the nickname for good subruns where the TPC, WCs,
and ToF is working properly. \cref{tab:samDatasetProps} shows the requirements
in the dataset definitions.

\begin{table}[!hbtp]
  \begin{center}
    \caption{SAM file selection for datasets.}
    \label{tab:samDatasetProps}
    \small
    \begin{tabu}{|p{2.5in}|p{2.4in}|} \hline
      SAM Cut & Explaination \\ \hline \hline
      create\_date $<$ `2017-06-02' & Don't include later processing \\ \hline
      data\_tier digits & Unpacked files \\ \hline
      data\_tier digits & Unpacked files \\ \hline
      lariat\_mid\_f\_mc7anb $<$ 0 or $>$ 0 & Positive or negative magnet polarity \\ \hline
      run\_number $>$= 8000 and run\_number $<$= 10226 & The Run II set of runs \\ \hline
      TPC\_voltages\_nominal & Self-explanatory \\ \hline
      defname: TPC\_MaxGainAndFilter & TPC preamp/shapers set to maximum gain and shaping time (3 \& 3)\\ \hline
      defname: TPC\_nominal\_read\_out\_and\_timing & Normal TPC readout mode \\ \hline
      defname: BothTOF\_OnAndReadOut & Self-explanatory \\ \hline
      defname: AllMWPC\_OnAndReadOut & Self-explanatory \\ \hline
    \end{tabu}
  \end{center}
\end{table}

Additional dataset definitions are for specific secondary beam momenta and
tertiary magnet currents. These are subsets of
\texttt{Lovely1\_Pos\_RunII\_elenag\_v04}. For the nominal momentum and
current, these datasets require the \texttt{secondary.momentum} and
\texttt{tertiary.magnet\_current} to be within one unit of the nominal value in
GeV and A, respectively. \cref{tab:samSpecificDatasets} shows the dataset
definition names. There are no files for 64 GeV secondary beam with magnet
currents between 0 and 59 A. There are 544 files with secondary beam energy
less than 63~GeV and only 189 files for secondary beam energy less than 59 GeV.
There are no files with secondary beam energy greater than 65~GeV.
\cref{fig:datasetConditions} shows the number of files per magnet current,
compared between data in the Run Summary database and SAM database.

% 64 GeV with current < 39: 2787 files
% 64 GeV with current < 19: 2786 files
% 64 GeV with current > 5 <= 11: 0 files
% 64 GeV with current > 0 <= 5: 0 files
% 64 GeV with current = 0: 2786 files
% 64 GeV with current >= 39 & < 59: 0 files
% 64 GeV with current > 61 & < 99: 2 files
% 64 GeV with current > 101: 2 files
% 64 GeV with current >= 39 <= 41: 0 files
% 64 GeV with current >= 19 <= 21: 0 files

\begin{table}[!hbtp]
  \begin{center}
    \caption{SAM dataset definitions for specific secondary beam momenta (in GeV) 
                and tertiary magnet current (in A) and the number of associated files.}
    \label{tab:samSpecificDatasets}
    \small
    \begin{tabu}{|l|l|} \hline
      SAM Dataset Definition Name & Number of Files \\ \hline \hline
      Lovely1\_Pos\_RunII\_jhugon\_current100\_secondary64\_v1   & 36797 \\ \hline
      Lovely1\_Pos\_RunII\_jhugon\_current60\_secondary64\_v1    & 17078 \\ \hline
%      Lovely1\_Pos\_RunII\_jhugon\_current40\_secondary64\_v1    & 0     \\ \hline
%      Lovely1\_Pos\_RunII\_jhugon\_current20\_secondary64\_v1    & 0     \\ \hline
%      Lovely1\_Pos\_RunII\_jhugon\_currentLT19\_secondary64\_v1  & 2786  \\ \hline
      Lovely1\_Pos\_RunII\_jhugon\_current0\_secondary64\_v1     & 2786  \\ \hline
      Lovely1\_Neg\_RunII\_jhugon\_current100\_secondary64\_v1   & 14342 \\ \hline
      Lovely1\_Neg\_RunII\_jhugon\_current60\_secondary64\_v1    & 35717 \\ \hline
    \end{tabu}
  \end{center}
\end{table}

\begin{figure}[!hbtp]
  \begin{center}
    \subfloat[]{\label{fig:datasetConditions_hist}
          \includegraphics[width=0.48\textwidth]{figures/beamConditions/currentHist.png}
    }
    \subfloat[]{\label{fig:datasetConditions_hist2d}
          \includegraphics[width=0.48\textwidth]{figures/beamConditions/currentHist2D.png}
    }
    \\
    \subfloat[]{\label{fig:datasetConditions_currentVrun}
          \includegraphics[width=0.48\textwidth]{figures/beamConditions/currentVrun.png}
    }
    \caption{%
                Summary plots of the magnet current during Run II. A histogram of 
                magnet current for ``lovely'', positive polarity subruns is 
                shown in \subref{fig:datasetConditions_hist}, the magnet current 
                from the run summary v. SAM is shown in \subref{fig:datasetConditions_hist2d}, 
                and the magnet current v. run number is shown in 
                \subref{fig:datasetConditions_currentVrun}, where the run summary data 
                includes negative polarity and non-lovely runs.
            }
    \label{fig:datasetConditions}
  \end{center}
\end{figure}


The data and MC samples are processed using version v06\_34\_00 of \texttt{larsoft} and
\texttt{lariatsoft}. This software is modified with the git branches listed in
\cref{tab:software}. The web page: 
\mbox{\url{https://github.com/jhugon/lariatPionAbs/blob/master/README.md}}
has instructions for installing the various software versions.

\begin{table}[!hbtp]
  \begin{center}
    \caption{Git branch or tag names used for software packages used in this study.}
    \label{tab:software}
    \begin{tabu}{|l|l|} \hline
      \texttt{larsoft} Package & Branch or Tag \\ \hline \hline
      \texttt{lariatsoft} & feature/jhugon\_PionAbsAndChEx \\ \hline
      \texttt{larana} & feature/jhugon\_likelihoodPID\ \\ \hline
      \texttt{larreco} & feature/jhugon\_caloTruth \\ \hline
      \texttt{lardataobj} & feature/jhugon\_caloTruth \\ \hline
    \end{tabu}
  \end{center}
\end{table}

The TPC reconstruction uses the algorithms listed in \cref{tab:recoAlgs}.

\begin{table}[!hbtp]
  \begin{center}
    \caption{TPC reconstruction algorithms used in this analysis.}
    \label{tab:recoAlgs}
    \begin{tabu}{|l|l|l|} \hline
      Reconstruction Stage & Algorithm & Parameters\\ \hline \hline
      Hit finding & \texttt{gaushit} & default \\ \hline
      2D clustering & \texttt{trajcluster} & default \\ \hline
      3D track finding & \texttt{pmtrack} & default \\ \hline
      Calorimetry & \texttt{calo} & default \\ \hline
    \end{tabu}
  \end{center}
\end{table}

Monte Carlo data is generated using a single particle gun starting from the
closest wire chamber to the TPC. The particle gun starting position and angle
are generated from uniform distributions with boundaries similar to that found
in data. The momentum distribution is flat from 0 to \SI{1.5}{\GeVc{}}. The MC
for each particle species is then reweighted based on the momentum distribution
observed in the beamline MC. The ratios of particle species in the beamline MC
are also used to weight the single particle gun MC.

\section{Event Selection}

In this analysis, particles that create hits in the WC, ToF, and TPC are known
as ``primary particles.'' The particles created in interactions of primary
particles are called ``secondary particles.'' These terms also lead to
``primary tracks'' and ``secondary tracks'' that are TPC tracks reconstructed
from primary particles and secondary particles, respectively.

\subsection{Primary \pip{} Selection}

The first part of the selection is meant to identify a pure sample of primary
tracks originating from primary \pip{} particles and called ``primary \pip{}
selection''. The cuts are shown in \cref{tab:pipCuts}. Wire chamber tracks must
pass the ``picky track'' cuts, which means that there is exactly one hit on
each X and Y-plane in each of the four WC stations. This is required to ensure
the purest sample of tracks with the best momentum resolution. Since a WC track
is required, this means a single hit is required in each WC plane. The WC
momentum is required to be between \SI{100}{\MeVc{}} and \SI{1100}{\MeVc{}},
and the projected WC track direction must pass through the TPC flange. 

\begin{table}[!hbtp]
  \begin{center}
    \small
    \caption{Primary \pip{} selection cuts. See \cref{tab:WCTPCMatching} for WC-TPC track matching.}
    \label{tab:pipCuts}
    \begin{tabu}{|l|l|} \hline
      Variable & Cut \\ \hline \hline
      N$_{\text{hits}}$ in each WC plane & 1 \\ \hline
      WC momentum & \SIrange{100}{1100}{\MeVc{}} \\ \hline
      x, y of WC track projected to $z=0$ & $\sqrt{(x-\SI{23.75}{\cm})^2+(y-\SI{0.2}{\cm})^2} < \SI{11.93}{\cm}$ \\ \hline
      First ToF & \SIrange{0}{25}{\nano \second} \\ \hline
      Primary track start $z$ & $< \SI{2}{\cm}$ \\ \hline
      \sout{N$_{\text{tracks}}$ in  $z<\SI{14}{\cm}$} & \sout{$< 4$} \\ \hline
      \sout{N$_{\text{tracks}}$ with length $<\SI{5}{\cm}$} & \sout{$< 3$} \\ \hline
      N TPC tracks matched to the WC track & 1 \\ \hline
    \end{tabu}
  \end{center}
\end{table}

The particle mass squared reconstructed using WC and ToF information, \mSqrBeam{}, is used to
reduce kaon and proton contamination. The mass squared is calculated as:

\begin{equation}
\mSqrBeam{} = p^2 \left( \frac{t^2 c^2}{d^2} - 1 \right),
\end{equation}

where $p$ is the WC momentum in \si{\MeVc{}}, $t$ is the ToF in \si{\nano
\second}, $d$ is the distance a particle travels between the ToF counters,
\SI{6.686}{\meter}, and $c$ is the speed of light in \si[per-mode =
symbol]{\meter \per \nano \second}.\footnote{The time for a zero mass particle
to go through the ToF counters, $d/c$, is \SI{22.3}{\nano \second}}. The mass
squared is used instead of the mass, because the finite resolution of the ToF
makes some particles appear to go faster than the speed of light, making the
mass undefined ($t < d/c$ and thus $\mSqrBeam{} < 0$). To reject kaons and
pions, \mSqrBeam{} must be less than $(223.6)^2 = $ \SI{5e4}{(\MeVcc{}
)\squared}.

To reduce the number of events from \ep{} and beam halo muons as well as
high-occupancy events, some additional cuts are made on TPC tracks. The primary
track must start in the first \SI{2}{\cm{}} of the TPC. \sout{There must be
less than 4 tracks in the first \SI{14}{\cm}. Also, to reduce the \ep{}
contamination, less than 3 tracks with lengths less than \SI{5}{\cm} are
required.} %end sout

Finally, the WC track must be matched to exactly one TPC track. The cuts are
summarized in \cref{tab:WCTPCMatching}. The matching begins by only looking at
TPC tracks that start in the first \SI{2}{\cm} of the TPC in $z$, and go
through the TPC flange (like the WC tracks). Then all of those TPC tracks and
the WC track are projected to the front face of the TPC (the $z=0$ plane) to
find $x$ and $y$-coordinates. For a WC track and TPC track to match, the
difference in the $x$-coordinates, $\Delta x$, must be between \SI{-4}{\cm} and
\SI{6}{\cm}, and the difference in the $y$-coordinates, $\Delta y$, must be
between \SI{-5}{\cm} and \SI{5}{\cm}. The difference in the 3D angle between
the WC track and TPC track, $\Delta \alpha$, must be less than
\SI{10}{\degree}.

\begin{table}[!hbtp]
  \begin{center}
    \small
    \caption{Wire chamber track to TPC track matching cuts.}
    \label{tab:WCTPCMatching}
    \begin{tabu}{|l|l|} \hline
      Variable & Cut \\ \hline \hline
      TPC track start or end $z$ & $<\SI{2}{\cm} \\ \hline
      x, y of TPC track projected to $z=0$ & $\sqrt{(x-\SI{23.75}{\cm})^2+(y-\SI{0.2}{\cm})^2} < \SI{11.93}{\cm}$ \\ \hline
      Tracks projected to $z=0$: $\Delta x$  & Data: \SIrange{-4}{6}{\cm}, MC: \SIrange{-5}{5}{\cm} \\ \hline
      Tracks projected to $z=0$: $\Delta y$  & \SIrange{-5}{5}{\cm} \\ \hline
      3D angle: $\Delta \alpha$  & $<\SI{10}{\degree}$ \\ \hline
    \end{tabu}
  \end{center}
\end{table}

Since the MC samples are generated using a single particle gun, there is no
accompanying WC or ToF information. The assumed WC momentum and direction are
set equal to the true initial momentum and direction of the generated primary
particle. Momentum cuts and WC-TPC track matching are then performed in MC just
as in data, using the true initial particle information on position, direction,
and momentum as the WC values. The $\Delta x$ is required to be between
\SI{-5}{\cm} and \SI{5}{\cm}, which differs from the data. This is thought to
be some kind of timing offset in the data that is not present in the MC. The
TPC track cuts are applied just as in data.

\subsection{Signal Definition}

Pions that have true inelastic interactions in the fiducial region with no
charged pions in the final state. Also, the reconstructed interaction vertex
must be withing \SI{1}{\cm} of the true interaction vertex. The reconstructed
interation vertex is defined as the end of the primary track.

FIDUCIAL INTERACTION REGION DEFINITION

\subsection{Signal Selction}

RECO INTERACTION IN FIDUCIAL, SECONDARY TRACK DEFINITION, SECONDARY SELECTION

\section{Cross-section Calculation}

\section{Systematic Uncertainties}

\section{Conclusions}

\appendix
\section{Event Displays}

% Setup listings
\lstset{%
    basicstyle=\scriptsize\ttfamily,
    %frame=single,
    breaklines=true
}

\begin{figure}[!hbtp]
  \begin{center}
    \subfloat[]{\label{fig:evd_pipMC_208638_twq}
        \includegraphics[width=0.48\textwidth]{figures/evd/evd-twq-proj-1-208638.pdf}
    }
    \subfloat[]{\label{fig:evd_pipMC_208638_ortho}
        \includegraphics[width=0.48\textwidth]{figures/evd/evd-larortho3d-1-208638.pdf}
    }
\\
\begin{lstlisting}
process: primary, endProcess: pi+Inelastic, trueEndProcess: 13, nDaughters: 7, trueNSecondaryChPions: 1, trueNSecondaryPiZeros: 0, trueNSecondaryProtons: 0, End: 20.42, -1.262, 62.98
TrackId: 1 MotherId: 0 PDG 211 KE 1263.1 len: 159.1 start: 26.3, -1.3, -96.0 end: 20.4, -1.3, 63.0 Process: primary EndProcess: pi+Inelastic
TrackId: 34 MotherId: 1 PDG 211 KE 1021.4 len: 969.7 start: 20.4, -1.3, 63.0 end: 221.2, -224.4, 985.0 Process: pi+Inelastic EndProcess: CoupledTransportation
TrackId: 36 MotherId: 1 PDG 22 KE 6.9 len: 62.5 start: 20.4, -1.3, 63.0 end: 13.3, -36.0, 67.0 Process: pi+Inelastic EndProcess: phot
Primary Track start point: (22.3,-1.2,0.6)
Primary Track end point: (20.5,-1.0,63.7)
Secondary Track Start: (20.7,-1.2,64.4) End: (26.0,-7.5,90.1)
\end{lstlisting}
    \caption{%
                Event display of \pip{} MC event 208638. 
                Time v. wire is shown in \subref{fig:evd_pipMC_208638_twq}, 
                an orthographic projection in \subref{fig:evd_pipMC_208638_ortho},
                and truth and reconstruction information is also listed.
            }
    \label{fig:evd_pipMC_208638}
  \end{center}
\end{figure}

\begin{figure}[!hbtp]
  \begin{center}
    \subfloat[]{\label{fig:evd_pipMC_208639_twq}
        \includegraphics[width=0.48\textwidth]{figures/evd/evd-twq-proj-1-208639.pdf}
    }
    \subfloat[]{\label{fig:evd_pipMC_208639_ortho}
        \includegraphics[width=0.48\textwidth]{figures/evd/evd-larortho3d-1-208639.pdf}
    }
\\
\begin{lstlisting}
process: primary, endProcess: Decay, trueEndProcess: 6, nDaughters: 2, trueNSecondaryChPions: 0, trueNSecondaryPiZeros: 0, trueNSecondaryProtons: 0, End: 36.39, -4.516, 199.5
TrackId: 1 MotherId: 0 PDG 211 KE 623.1 len: 296.0 start: 30.1, 1.8, -96.0 end: 36.4, -4.5, 199.5 Process: primary EndProcess: Decay
Primary Track start point: (29.4,2.0,0.2)
Primary Track end point: (35.5,0.0,90.0)
\end{lstlisting}
    \caption{%
                Event display of \pip{} MC event 208639. 
                Time v. wire is shown in \subref{fig:evd_pipMC_208639_twq}, 
                and an orthographic projection in \subref{fig:evd_pipMC_208639_ortho}
                and truth and reconstruction information is also listed.
            }
    \label{fig:evd_pipMC_208639}
  \end{center}
\end{figure}

\begin{figure}[!hbtp]
  \begin{center}
    \subfloat[]{\label{fig:evd_pipMC_208641_twq}
        \includegraphics[width=0.48\textwidth]{figures/evd/evd-twq-proj-1-208641.pdf}
    }
    \subfloat[]{\label{fig:evd_pipMC_208641_ortho}
        \includegraphics[width=0.48\textwidth]{figures/evd/evd-larortho3d-1-208641.pdf}
    }
\\
\begin{lstlisting}
process: primary, endProcess: pi+Inelastic, trueEndProcess: 13, nDaughters: 7, trueNSecondaryChPions: 0, trueNSecondaryPiZeros: 1, trueNSecondaryProtons: 1, End: 23.57, -9.378, 21.69
TrackId: 1 MotherId: 0 PDG 211 KE 466.5 len: 117.7 start: 22.6, -7.1, -96.0 end: 23.6, -9.4, 21.7 Process: primary EndProcess: pi+Inelastic
TrackId: 16 MotherId: 1 PDG 111 KE 219.3 len: 0.0 start: 23.6, -9.4, 21.7 end: 23.6, -9.4, 21.7 Process: pi+Inelastic EndProcess: Decay
TrackId: 17 MotherId: 1 PDG 2212 KE 146.1 len: 15.2 start: 23.6, -9.4, 21.7 end: 18.8, -13.1, 35.7 Process: pi+Inelastic EndProcess: LArVoxelReadoutScoringProcess
TrackId: 32 MotherId: 16 PDG 22 KE 295.0 len: 17.1 start: 23.6, -9.4, 21.7 end: 33.0, -23.5, 24.2 Process: Decay EndProcess: conv
TrackId: 33 MotherId: 16 PDG 22 KE 59.2 len: 25.3 start: 23.6, -9.4, 21.7 end: 14.3, -31.7, 14.0 Process: Decay EndProcess: conv
Primary Track start point: (23.7,-8.1,0.4)
Primary Track end point: (23.5,-9.4,21.9)
Secondary Track Start: (23.4,-9.4,22.2) End: (18.7,-13.1,35.8)
\end{lstlisting}
    \caption{%
                Event display of \pip{} MC event 208641. 
                Time v. wire is shown in \subref{fig:evd_pipMC_208641_twq}, 
                and an orthographic projection in \subref{fig:evd_pipMC_208641_ortho},
                and truth and reconstruction information is also listed.
            }
    \label{fig:evd_pipMC_208641}
  \end{center}
\end{figure}

\begin{figure}[!hbtp]
  \begin{center}
    \subfloat[]{\label{fig:evd_pipMC_208643_twq}
        \includegraphics[width=0.48\textwidth]{figures/evd/evd-twq-proj-1-208643.pdf}
    }
    \subfloat[]{\label{fig:evd_pipMC_208643_ortho}
        \includegraphics[width=0.48\textwidth]{figures/evd/evd-larortho3d-1-208643.pdf}
    }
\\
\begin{lstlisting}
process: primary, endProcess: pi+Inelastic, trueEndProcess: 13, nDaughters: 13, trueNSecondaryChPions: 2, trueNSecondaryPiZeros: 0, trueNSecondaryProtons: 0, End: 51.82, -3.928, 75.39
TrackId: 1 MotherId: 0 PDG 211 KE 925.5 len: 172.5 start: 35.8, 0.5, -96.0 end: 51.8, -3.9, 75.4 Process: primary EndProcess: pi+Inelastic
TrackId: 38 MotherId: 1 PDG 211 KE 209.3 len: 16.6 start: 51.8, -3.9, 75.4 end: 48.3, 7.7, 86.6 Process: pi+Inelastic EndProcess: pi+Inelastic
TrackId: 39 MotherId: 1 PDG 211 KE 203.3 len: 46.4 start: 51.8, -3.9, 75.4 end: 59.5, -48.8, 83.9 Process: pi+Inelastic EndProcess: LArVoxelReadoutScoringProcess
TrackId: 42 MotherId: 1 PDG 22 KE 5.6 len: 55.1 start: 51.8, -3.9, 75.4 end: 39.1, -56.4, 86.2 Process: pi+Inelastic EndProcess: conv
TrackId: 45 MotherId: 1 PDG 22 KE 0.9 len: 56.0 start: 51.8, -3.9, 75.4 end: 26.8, 10.5, 85.3 Process: pi+Inelastic EndProcess: phot
TrackId: 47 MotherId: 1 PDG 22 KE 0.4 len: 99.0 start: 51.8, -3.9, 75.4 end: 30.2, -19.2, 63.0 Process: pi+Inelastic EndProcess: phot
\end{lstlisting}
    \caption{%
                Event display of \pip{} MC event 208643. 
                Time v. wire is shown in \subref{fig:evd_pipMC_208643_twq}, 
                and an orthographic projection in \subref{fig:evd_pipMC_208643_ortho},
                and truth and reconstruction information is also listed.
            }
    \label{fig:evd_pipMC_208643}
  \end{center}
\end{figure}

\begin{figure}[!hbtp]
  \begin{center}
    \subfloat[]{\label{fig:evd_pipMC_208638_twq}
        \includegraphics[width=0.48\textwidth]{figures/evd/evd-twq-proj-1-208644.pdf}
    }
    \subfloat[]{\label{fig:evd_pipMC_208644_ortho}
        \includegraphics[width=0.48\textwidth]{figures/evd/evd-larortho3d-1-208644.pdf}
    }
\\
\begin{lstlisting}
process: primary, endProcess: CoupledTransportation, trueEndProcess: 15, nDaughters: 2, trueNSecondaryChPions: 0, trueNSecondaryPiZeros: 0, trueNSecondaryProtons: 0, End: -79.57, 9.515, 1010
TrackId: 1 MotherId: 0 PDG 211 KE 1334.0 len: 1111.6 start: 26.0, -0.4, -96.0 end: -79.6, 9.5, 1010.4 Process: primary EndProcess: CoupledTransportation
Primary Track start point: (15.6,1.4,1.4)
Primary Track end point: (3.7,2.1,89.2)
Secondary Track Start: (11.3,2.2,35.1) End: (11.8,2.8,38.0)
Secondary Track Start: (11.3,-5.3,31.6) End: (11.0,4.7,26.8)
Secondary Track Start: (10.6,0.8,31.8) End: (11.0,-28.3,46.3)
Secondary Track Start: (4.8,2.0,83.1) End: (5.7,1.7,89.0)
Secondary Track Start: (12.3,2.1,20.0) End: (9.6,3.1,22.6)
\end{lstlisting}
    \caption{%
                Event display of \pip{} MC event 208644. 
                Time v. wire is shown in \subref{fig:evd_pipMC_208644_twq}, 
                and an orthographic projection in \subref{fig:evd_pipMC_208644_ortho},
                and truth and reconstruction information is also listed.
            }
    \label{fig:evd_pipMC_208644}
  \end{center}
\end{figure}

\begin{figure}[!hbtp]
  \begin{center}
    \subfloat[]{\label{fig:evd_pipMC_208638_twq}
        \includegraphics[width=0.48\textwidth]{figures/evd/evd-twq-proj-1-208645.pdf}
    }
    \subfloat[]{\label{fig:evd_pipMC_208645_ortho}
        \includegraphics[width=0.48\textwidth]{figures/evd/evd-larortho3d-1-208645.pdf}
    }
\\
\begin{lstlisting}
process: primary, endProcess: pi+Inelastic, trueEndProcess: 13, nDaughters: 8, trueNSecondaryChPions: 2, trueNSecondaryPiZeros: 0, trueNSecondaryProtons: 0, End: 24.49, 0.2961, -7
TrackId: 1 MotherId: 0 PDG 211 KE 1350.8 len: 89.2 start: 30.2, 1.2, -96.0 end: 24.5, 0.3, -7.0 Process: primary EndProcess: pi+Inelastic
TrackId: 8 MotherId: 1 PDG 211 KE 670.6 len: 57.0 start: 24.5, 0.3, -7.0 end: 18.5, -15.9, 46.9 Process: pi+Inelastic EndProcess: pi+Inelastic
TrackId: 9 MotherId: 1 PDG 211 KE 375.8 len: 360.2 start: 24.5, 0.3, -7.0 end: -175.5, 224.4, 191.0 Process: pi+Inelastic EndProcess: CoupledTransportation
TrackId: 127 MotherId: 8 PDG 2212 KE 356.2 len: 507.5 start: 18.5, -15.9, 46.9 end: -111.2, -224.4, 490.9 Process: pi+Inelastic EndProcess: CoupledTransportation
TrackId: 129 MotherId: 8 PDG 2212 KE 33.0 len: 1.1 start: 18.5, -15.9, 46.9 end: 18.9, -16.4, 47.8 Process: pi+Inelastic EndProcess: LArVoxelReadoutScoringProcess
Primary Track start point: (22.1,-2.3,0.3)
Primary Track end point: (17.5,-8.2,54.3)
Secondary Track Start: (18.1,6.0,0.2) End: (2.5,20.3,16.3)
Secondary Track Start: (17.5,-17.3,49.6) End: (15.5,-20.0,54.9)
\end{lstlisting}
    \caption{%
                Event display of \pip{} MC event 208645. 
                Time v. wire is shown in \subref{fig:evd_pipMC_208645_twq}, 
                and an orthographic projection in \subref{fig:evd_pipMC_208645_ortho},
                and truth and reconstruction information is also listed.
            }
    \label{fig:evd_pipMC_208645}
  \end{center}
\end{figure}

\begin{figure}[!hbtp]
  \begin{center}
    \subfloat[]{\label{fig:evd_pipMC_208638_twq}
        \includegraphics[width=0.48\textwidth]{figures/evd/evd-twq-proj-1-208646.pdf}
    }
    \subfloat[]{\label{fig:evd_pipMC_208646_ortho}
        \includegraphics[width=0.48\textwidth]{figures/evd/evd-larortho3d-1-208646.pdf}
    }
\\
\begin{lstlisting}
process: primary, endProcess: pi+Inelastic, trueEndProcess: 13, nDaughters: 12, trueNSecondaryChPions: 2, trueNSecondaryPiZeros: 0, trueNSecondaryProtons: 1, End: 25.46, 5.16, 14.29
TrackId: 1 MotherId: 0 PDG 211 KE 619.5 len: 110.3 start: 26.9, 2.7, -96.0 end: 25.5, 5.2, 14.3 Process: primary EndProcess: pi+Inelastic
TrackId: 17 MotherId: 1 PDG 211 KE 181.6 len: 16.8 start: 25.5, 5.2, 14.3 end: 14.0, 3.2, 26.5 Process: pi+Inelastic EndProcess: pi+Inelastic
TrackId: 18 MotherId: 1 PDG -211 KE 100.4 len: 20.1 start: 25.5, 5.2, 14.3 end: 16.5, -5.1, -0.5 Process: pi+Inelastic EndProcess: pi-Inelastic
TrackId: 19 MotherId: 1 PDG 2212 KE 50.2 len: 2.3 start: 25.5, 5.2, 14.3 end: 26.0, 5.3, 16.5 Process: pi+Inelastic EndProcess: LArVoxelReadoutScoringProcess
TrackId: 34 MotherId: 18 PDG 2212 KE 99.7 len: 7.0 start: 16.5, -5.1, -0.5 end: 12.0, -2.0, -4.9 Process: pi-Inelastic EndProcess: LArVoxelReadoutScoringProcess
TrackId: 71 MotherId: 17 PDG 211 KE 91.5 len: 23.1 start: 14.0, 3.2, 26.5 end: 12.5, 15.3, 7.5 Process: pi+Inelastic EndProcess: LArVoxelReadoutScoringProcess
TrackId: 89 MotherId: 71 PDG 14 KE 29.8 len: 278.4 start: 12.5, 15.3, 7.5 end: 122.0, -224.4, 97.1 Process: Decay EndProcess: CoupledTransportation
TrackId: 90 MotherId: 71 PDG -13 KE 4.1 len: 0.1 start: 12.5, 15.3, 7.5 end: 12.4, 15.4, 7.4 Process: Decay EndProcess: LArVoxelReadoutScoringProcess
TrackId: 110 MotherId: 90 PDG -14 KE 33.7 len: 265.6 start: 12.4, 15.4, 7.4 end: 118.2, -224.4, -35.2 Process: Decay EndProcess: CoupledTransportation
TrackId: 111 MotherId: 90 PDG 12 KE 24.3 len: 361.3 start: 12.4, 15.4, 7.4 end: 347.7, 25.9, 141.5 Process: Decay EndProcess: CoupledTransportation
TrackId: 112 MotherId: 90 PDG -11 KE 46.7 len: 16.3 start: 12.4, 15.4, 7.4 end: -0.8, 23.6, 7.0 Process: Decay EndProcess: annihil
Primary Track start point: (26.0,4.7,-0.1)
Primary Track end point: (25.5,5.1,12.2)
Secondary Track Start: (14.3,3.3,26.2) End: (12.5,15.5,7.5)
Secondary Track Start: (25.5,4.9,13.8) End: (18.2,-3.6,2.0)
Secondary Track Start: (25.4,5.1,14.4) End: (15.0,3.4,25.5)
Secondary Track Start: (12.3,15.7,7.5) End: (8.8,19.6,7.5)
Secondary Track Start: (25.7,5.2,14.9) End: (26.1,5.3,16.7)
Secondary Track Start: (17.7,-4.0,1.3) End: (16.9,-4.8,0.0)
\end{lstlisting}
    \caption{%
                Event display of \pip{} MC event 208646. 
                Time v. wire is shown in \subref{fig:evd_pipMC_208646_twq}, 
                and an orthographic projection in \subref{fig:evd_pipMC_208646_ortho},
                and truth and reconstruction information is also listed.
            }
    \label{fig:evd_pipMC_208646}
  \end{center}
\end{figure}

\begin{figure}[!hbtp]
  \begin{center}
    \subfloat[]{\label{fig:evd_pipMC_208647_twq}
        \includegraphics[width=0.48\textwidth]{figures/evd/evd-twq-proj-1-208647.pdf}
    }
    \subfloat[]{\label{fig:evd_pipMC_208647_ortho}
        \includegraphics[width=0.48\textwidth]{figures/evd/evd-larortho3d-1-208647.pdf}
    }
\\
\begin{lstlisting}
process: primary, endProcess: pi+Inelastic, trueEndProcess: 13, nDaughters: 6, trueNSecondaryChPions: 1, trueNSecondaryPiZeros: 0, trueNSecondaryProtons: 0, End: 15.36, -3.344, 63.95
TrackId: 1 MotherId: 0 PDG 211 KE 862.4 len: 160.6 start: 28.5, 1.4, -96.0 end: 15.4, -3.3, 63.9 Process: primary EndProcess: pi+Inelastic
TrackId: 28 MotherId: 1 PDG 211 KE 209.9 len: 69.8 start: 15.4, -3.3, 63.9 end: 19.0, -43.1, 7.0 Process: pi+Inelastic EndProcess: LArVoxelReadoutScoringProcess
TrackId: 62 MotherId: 28 PDG 14 KE 29.8 len: 188.7 start: 19.0, -43.1, 7.0 end: 60.1, -224.4, -25.1 Process: Decay EndProcess: CoupledTransportation
TrackId: 63 MotherId: 28 PDG -13 KE 4.1 len: 0.0 start: 19.0, -43.1, 7.0 end: 19.0, -43.1, 7.0 Process: Decay EndProcess: LArVoxelReadoutScoringProcess
TrackId: 64 MotherId: 63 PDG -14 KE 48.1 len: 239.8 start: 19.0, -43.1, 7.0 end: -82.7, -224.4, 126.6 Process: Decay EndProcess: CoupledTransportation
TrackId: 65 MotherId: 63 PDG 12 KE 22.0 len: 286.9 start: 19.0, -43.1, 7.0 end: -43.2, 224.4, -76.0 Process: Decay EndProcess: CoupledTransportation
TrackId: 66 MotherId: 63 PDG -11 KE 34.6 len: 0.7 start: 19.0, -43.1, 7.0 end: 19.5, -42.7, 6.9 Process: Decay EndProcess: LArVoxelReadoutScoringProcess
Primary Track start point: (20.8,-0.3,0.3)
Primary Track end point: (15.4,-4.1,64.2)
\end{lstlisting}
    \caption{%
                Event display of \pip{} MC event 208647. 
                Time v. wire is shown in \subref{fig:evd_pipMC_208647_twq}, 
                and an orthographic projection in \subref{fig:evd_pipMC_208647_ortho},
                and truth and reconstruction information is also listed.
            }
    \label{fig:evd_pipMC_208647}
  \end{center}
\end{figure}

\bibliographystyle{myEPJC}
\bibliography{mybib.bib}

\end{document}
