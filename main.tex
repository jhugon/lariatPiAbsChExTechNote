\documentclass[letterpaper,12pt]{article}

\usepackage{graphicx}
\DeclareGraphicsExtensions{.pdf,.png,.gif,.jpg}
\usepackage[usenames,dvipsnames,svgnames,table]{xcolor}
\definecolor{light-gray}{gray}{0.5}
\usepackage{amsmath}
\usepackage{amsthm}
\usepackage[hyperfootnotes=false]{hyperref}
\hypersetup{colorlinks=true,linkcolor=blue,anchorcolor=blue,citecolor=blue,filecolor=blue,urlcolor=blue,bookmarksnumbered=true,pdfview=FitB} %
\usepackage{doi}
\usepackage{tikz}
\usepackage{bm}
\usepackage{tabu}
\usepackage{subfigure}
\usepackage{listings}
\usepackage[noabbrev,capitalize]{cleveref}
\newcommand{\mc}[1]{\multicolumn{1}{c}{#1}}
\usepackage{siunitx}
\sisetup{
  %use option "per-mode = reciprocal-positive-first" on a command to make it use e.g. m s^{-1}
  %use option "per-mode = symbol" on a command to make it use e.g. m/s
  per-mode = reciprocal-positive-first,
  binary-units = true,
  range-phrase = {\text{--}},  % or { to }
  range-units = single,  % or repeat or brackets
  list-units = single,  % or repeat or brackets
  multi-part-units = single,  % or repeat or brackets
  separate-uncertainty = true, % use \pm instead of brackets
  mode = text,
  %detect-mode = true,
  %detect-shape = true,
  detect-family = true,
  detect-weight = true
}%
%%% Stuff to print git commit in footer
\usepackage[mark]{gitinfo2}

\title{Pion Absorption $+$ Charge Exchange Cross-section with LArIAT}
\author{Justin Hugon, William Metcalf, Andrew Olivier,\\Martin Tzanov, Derek Walker}

\begin{document}

\maketitle

\section{Introduction}

When neutrinos interact with argon nuclei, in LArTPCs like MicroBooNE, SBND,
and DUNE, the hadronic recoil consists of charged and neutral pions along with
protons and neutrons. Understanding how these hadrons interact in argon is
important for accurately reconstructing the initial neutrino kinematics.  One
of the main missions of LArIAT is to measure the cross-sections of charged
pions on argon. This note focuses specifically on the pion absorption and
charge exchange pion interactions.

In pion absorption, a charged pion interacts with a nucleus to produce non-pion
hadrons. Protons and neutrons are the most common products of pion absorption,
but deuterons and kaons can also be
produced~\cite{Kotlinski:1998vh,Rowntree:1999dp,Kotlinski:2000hp,Androic:2001tq}.
Neutrons are difficult to detect in a LArTPC because, as neutral particles,
they only produce ionization electrons when they interact with nuclei. Thus,
observations of pion absorption in argon rely on identification of protons,
deuterons, and very rarely kaons.

In pion charge exchange, a charged pion interacts with a nucleus so that the
pion's charge is added to the nucleus and a neutral pion is produced. Neutral
pions decay to a pair of photons, and the short neutral pion lifetime means
that they decay near the nucleus where they were produced. Thus, charge
exchange interactions are similar to pion absorption interactions, but with the
addition of two electromagnetic showers.

\section{Data \& Monte Carlo Samples}

Data for this study comes from LArIAT Run II which took place in 2016. For the
main analysis, only the positive magnet polarity runs are used. The data
considered are from the SAM dataset definitions
\texttt{Lovely1\_Pos\_RunII\_elenag\_v04}
\texttt{Lovely1\_Neg\_RunII\_elenag\_v04}. \cref{tab:samDatasetProps} shows the
requirements in the dataset definitions.

\begin{table}
  \begin{center}
    \caption{SAM file selection for datasets.}
    \label{tab:samDatasetProps}
    \small
    \begin{tabu}{|p{2.5in}|p{2.4in}|} \hline
      SAM Cut & Explaination \\ \hline \hline
      create\_date $<$ `2017-06-02' & Don't include later processing \\ \hline
      data\_tier digits & Unpacked files \\ \hline
      data\_tier digits & Unpacked files \\ \hline
      lariat\_mid\_f\_mc7anb $<$ 0 or $>$ 0 & Positive or negative magnet polarity \\ \hline
      run\_number $>$= 8000 and run\_number $<$= 10226 & The Run II set of runs \\ \hline
      TPC\_voltages\_nominal & Self-explanatory \\ \hline
      defname: TPC\_MaxGainAndFilter & TPC preamp/shapers set to maximum gain and shaping time (3 \& 3)\\ \hline
      defname: TPC\_nominal\_read\_out\_and\_timing & Normal TPC readout mode \\ \hline
      defname: BothTOF\_OnAndReadOut & Self-explanatory \\ \hline
      defname: AllMWPC\_OnAndReadOut & Self-explanatory \\ \hline
    \end{tabu}
  \end{center}
\end{table}

The data and MC samples are processed using version v06\_15\_00 of \texttt{larsoft} and
\texttt{lariatsoft}. This software is modified with the git branches listed in
\cref{tab:software}.

\begin{table}
  \begin{center}
    \begin{tabu}{|l|l|} \hline
      \texttt{larsoft} Package & Branch or Tag \\ \hline \hline
      \texttt{lariatsoft} & feature/jhugon\_PionAbsAndChEx \\ \hline
      \texttt{larana} & feature/jhugon\_likelihoodPID\_forlarsoftv06\_15\_00 \\ \hline
      \texttt{larreco} & feature/jhugon\_caloTruth \\ \hline
      \texttt{lardataobj} & feature/jhugon\_caloTruth \\ \hline
      \texttt{larbatch} & c30e15939360 (ups/product\_deps from v06\_15\_00)\\ \hline
    \end{tabu}
    \caption{Git branch or tag names used for software packages used in this study.}
    \label{tab:software}
  \end{center}
\end{table}

\bibliographystyle{myEPJC}
\bibliography{mybib.bib}

\end{document}
